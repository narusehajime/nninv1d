%#pdflatex Naruse_Esurf_2020.tex

This study implemented an inverse model that reconstructs the flow characteristics of turbidity currents from their deposits using a NN, and verified its effectiveness at the field scale. In this study, we assumed that turbidity currents occur from suspended sediment clouds, which flow down from the steep slope in a submarine canyon to a gently sloping basin plain. The inverse model attempts to reconstruct seven model input parameters (height and length of the initial suspended sediment cloud, sediment concentration of four grain size classes, and slope of the basin plain) from the thickness and grain size distribution of the turbidite deposited on the basin plain. The forward model using one-dimensional shallow water equations was used to produce training data sets with random conditions in prescribed ranges. The NN was trained using the generated data to develop the inverse model. Thereafter, the test data generated independently from the training data were analyzed to verify the performance of the inverse model.

As a result of the training and tests conducted on the inverse model, the following was found:

\begin{enumerate}

\item More than 2000 data sets were required for the training to avoid overlearning. An increase in the number of training data sets results in improved performance of the inverse model; however, the degree of improvement becomes smaller even if more than 3000 data sets.

\item The hydraulic conditions and basin slopes were precisely reconstructed from the test data sets. The thickness and grain size distribution of the turbidites deposited over a 10 km-long interval in a sedimentary basin were sufficient to reconstruct the flow conditions.

\item The inverse model of this study is quite robust to random errors in the input data. The addition of a normal random number with about the same magnitude of the standard deviation to the original data had little effect on the results of the inverse analysis.

\item Judging from the results of subsampling tests, the inversion of turbidity currents can be performed if an individual turbidite can be correlated over 10 km at approximately 1 km intervals. 

\end{enumerate}

These results imply that the inverse model of turbidity currents proposed in this study is promising for analyzing field-scale turbidites. This method is expected to be applied to actual turbidites in the future.